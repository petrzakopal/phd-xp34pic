\documentclass[a4paper, twoside, 11pt]{article}
% It is needed to use this command for automatic compilation in VSCode
% !TEX program = lualatexmk

%DOCUMENT, PREAMBLE AND MACROS DESIGNED FOR LuaLaTeX%
\newcommand{\fbar}{\FloatBarrier}

\usepackage{amsmath} %matematic package%
\usepackage{amssymb} %for miscellaneous mathematical symbols, first usage was for tick symbol in math mode \checkmark%
\usepackage{textcomp} %for miscellaneous symbols%
%\usepackage{times} %times font%
\usepackage{graphicx} %enhanced support for craphics%
%\usepackage{mathptmx} %use Times as default text font, and provide maths support%
\usepackage{cmap} %mapování znaků - vyhledávání v pdf%
\usepackage[english]{babel}%CZ%
\usepackage[utf8]{inputenc}%kódování%
\usepackage[T1]{fontenc}%kódování%
\usepackage{multirow}%Multirow table support
\usepackage{float}%Improves the interface for defining floating objects such as figures and tables%
\usepackage{wasysym} %for various glyphs, symbols%

\usepackage{setspace}%spacing% 
\onehalfspacing

\usepackage{hyperref}
\hypersetup{
    colorlinks=true, %pokud nechci definovat citecolor=black aby byly odkazy citací černé, tak dám colorlinks=false,%
    bookmarks=true,
    linkcolor=black,
    citecolor=black,
    urlcolor=black,
}

%when using LuaLaTex, defining Times Fonts from your system - it has to be named like this and inserted ttf file in the folder of your tex file%
\usepackage{fontspec}
\selectlanguage{czech}
\setmainfont[Ligatures=TeX,BoldFont={* Bold}] {Times New Roman}
                                
\setsansfont[Ligatures=TeX,BoldFont={* Bold}]{Times New Roman}
                                      
\setmonofont{Courier}
 
%\usepackage[italic]{mathastext} %for text in math environment, better looking times then



%for CITATIONS URL to work, it is not needed when you are not using URL label%
\usepackage{url}
\usepackage{csquotes}
\usepackage[style=iso-numeric, backend=biber, isbn=true, urldate=iso,seconds=true, date=terse, datezeros=true, language=czech]{biblatex}
\addbibresource{src/bib/zdroje.bib} % BIB resources to import
%\DeclareUrlCommand\url{\def\UrlLeft{<}\def\UrlRight{>} \urlstyle{tt}}



%\usepackage{biblatex}
%END for citations%

%changing bibliography font%
\renewcommand*{\bibfont}{\fontspec{Times New Roman}}

\usepackage{comment} %For comments%
\usepackage{pdfpages}%for pdf pages%
\usepackage{enumerate}%For lists%
\usepackage{enumitem}%For Custom Numbering Nested Lists%
\setlist[enumerate]{label*=\arabic*.} %setting Number. numbering in lists%
\usepackage{tikz} %For vector graphics%
\usepackage{circuitikz}%For schemes%
\usepackage{pgf} %Post script graphics for tikz%

%pouze funguje v PDFLaTeX%
%\usepackage{tgtermes}%na times font, jiný nefunguje s vyhledáváním a copy%

%
\usepackage{placeins}%% for \FloatBarrier command that blocks floating with htbp! go over \FloatBarrier
\usepackage{mathrsfs}%packagee for math symbols for Laplace, Z transform etc., usage \mathscr{Z}
\usepackage{upgreek}%for upgreek symbols, specified tau \uptau
\usepackage{physics}% for derivations \dd
%this works only when using PDFLaTeX%
\usepackage[list=true,listformat=simple]{subcaption}
\usepackage[figurename=Fig.,font=small,labelfont=it,textfont=it]{caption} %for renaming figures instead of renewcommand, small for 11pt default is 10pt as needed in word template
\usepackage[tablename=Tab.,font=small,labelfont=it,
            textfont=it]{caption} %for renaming tables instead of renewcommand
            

% List of abbreviations and symbols
% Original code author: Jakub Kučera

\usepackage[nonumberlist,nopostdot,section=subsection,numberedsection]{glossaries}
% section = subsection is for glossaries title to appear as a subsection, numberedsection adds the subsec number

\newglossary[slg]{symbolslist}{symbol}{ntn1}{List of symbols}
\newglossary[slg]{abbreviationslist}{abbreviation}{ntn2}{List of abbreviations}

\makeglossaries

% include files with definitions
% PZ definitions
\newglossaryentry{abbreviation:soc}{
                type=abbreviationslist,
                name={SoC},
                description={System on a chip}
}

\newglossaryentry{abbreviation:ip}{
                type=abbreviationslist,
                name={IP},
                description={Intellectual property}
}
\newglossaryentry{abbreviation:fpga}{
                type=abbreviationslist,
                name={FPGA},
                description={Field Programmable Gate Array}
}
\newglossaryentry{abbreviation:nr}{
                type=abbreviationslist,
                name={NR},
                description={Newton Raphson}
}
\newglossaryentry{abbreviation:rtl}{
                type=abbreviationslist,
                name={RTL},
                description={Register Transfer Level}
}
\newglossaryentry{abbreviation:fsm}{
                type=abbreviationslist,
                name={FSM},
                description={Finite State Machine}
}
\newglossaryentry{abbreviation:cordic}{
                type=abbreviationslist,
                name={CORDIC},
                description={Coordinate Rotation Digital Computer}
}
\newglossaryentry{abbreviation:lut}{
                type=abbreviationslist,
                name={LUT},
                description={Look Up Table}
}
\newglossaryentry{abbreviation:cpu}{
                type=abbreviationslist,
                name={CPU},
                description={Central Processing Unit}
}
\newglossaryentry{abbreviation:isa}{
                type=abbreviationslist,
                name={ISA},
                description={Instruction Set Architecture}
}
\newglossaryentry{abbreviation:foss}{
                type=abbreviationslist,
                name={FOSS},
                description={Free and Open-Source Software}
}
\newglossaryentry{abbreviation:she}{
                type=abbreviationslist,
                name={SHE},
                description={Selective Harmonic Elimination}
}
\newglossaryentry{abbreviation:vsi}{
                type=abbreviationslist,
                name={VSI},
                description={Voltage Source Inverter}
}
\newglossaryentry{abbreviation:dc}{
                type=abbreviationslist,
                name={DC},
                description={Direct Current}
}

\newglossaryentry{symbol:Pn}{
    type=symbolslist, % glossary
    name=$P_\text{n}$, % jméno v seznamu
    description={jmenovitý výkon stroje}, %popis
    symbol = (W),
    sort=P % seředit podle
}


\newglossarystyle{myStyleAbbreviations}{
 \renewenvironment{theglossary}%
     {\begin{longtable}[l]{llp{\glsdescwidth}p{\glspagelistwidth}}}%
     {\end{longtable}}%
  \renewcommand*{\glossaryheader}{}%
  \renewcommand*{\glsgroupheading}[1]{}%
  \renewcommand*{\glossaryentryfield}[5]{%
   \bfseries \glstarget{##1}{##2} & ##3\glspostdescription\space ##5\\}%
  \renewcommand*{\glossarysubentryfield}[6]{%
     & \glstarget{##2}{\strut}##4\glspostdescription\space ##6\\}%
  \renewcommand*{\glsgroupskip}{}%  Pokud chci seskupovat podle abeced: \renewcommand*{\glsgroupskip}{ & \\}
}


\newglossarystyle{myStyleSymbols}{
  \renewenvironment{theglossary}%
    {\begin{longtable}[l]{llp{\glsdescwidth}p{\glspagelistwidth}}}%
    {\end{longtable}}%
  \renewcommand*{\glossaryheader}{}%
  \renewcommand*{\glsgroupheading}[1]{}%
  \renewcommand{\glossentry}[2]{%
    \glsentryitem{##1} \glstarget{##1}{\glossentryname{##1}} &
    \glossentrysymbol{##1} &
    \glossentrydesc{##1} &
    ##2\tabularnewline
  }%
  \renewcommand{\subglossentry}[3]{%
     &
     \glssubentryitem{##2}%
     \glossentrysymbol{##2} &
     \glstarget{##2}{\strut}\glossentrydesc{##2} & ##3\tabularnewline
  }%
  \renewcommand*{\glsgroupskip}{%
   }% Pokud chci seskupovat podel abecedy  \ifglsnogroupskip\else & & &\tabularnewline\fi
}
\renewcommand{\glossarypreamble}{\vspace*{-\baselineskip}} % deleting line after glossaries title

            %this works with LuaLaTex and fontspec package%
 \DeclareCaptionFont{times}{\fontspec{Times New Roman Italic}}

%labelfont and textfont defined here only works with previous declarecaptionfont times and fontspec%
\captionsetup{labelfont=times, textfont=times, labelsep=space}%no separator in captions
%

%\bibliographystyle{czechiso} %czechiso.bst in folder is needed for this style to work, available at http://www.fit.vutbr.cz/~martinek/latex/czechiso.html%

%\hyperref[label]{text}% Help for targeting labels

\usepackage{chngcntr} %for numbered figures with sections
\usepackage{tocloft}%better TOC

%\usepackage{a4wide}%širší a4%
\usepackage[inner=3cm,outer=2cm,top=2.5cm,bottom=2.5cm,footskip=1cm]{geometry}%for propper margins
\usepackage{textcase}%for making text uppercase without caps \MakeTextUppercase
 
 
\usepackage{titlesec}%for spacing text after sections
\usepackage{parskip}[]%for working \parskip
\newcommand{\sectionbreak}{\clearpage}%maybe for SECTIONS on a new page

\usepackage[titletoc]{appendix}%For appendix - přílohy, titletoc is crucial
%\renewcommand{\appendixname}{Příloha}

\setlength{\parindent}{0.5cm}%setting indent of paragraph to 0.5cm
\setlength{\parskip}{0em}%setting parskip to 0 for \titleformat to work properly with parskip package

\usepackage{colortbl}%for colored cells
\usepackage{xcolor}%for colors
\definecolor{ctublue}{HTML}{0065BD}%defining ctu color
\definecolor{ctugreen}{HTML}{A2AD00}
\definecolor{ctured}{HTML}{C60C30}
\definecolor{ctuyellow}{HTML}{F0AB00}
\definecolor{ctugreenyblue}{HTML}{00B2A9}
\definecolor{ctulightblue}{HTML}{6AADE4}
\definecolor{ctuorange}{HTML}{E05206}
\definecolor{lightgray}{HTML}{D1D5DB}
\definecolor{codeblue}{HTML}{D9E2F3}


\titlespacing*{\section}{0em}{1em}{-\parskip}%spacing text after sections from titlesec package
\titlespacing*{\subsection}{0em}{1em}{-\parskip}%spacing text after sections from titlesec package
\titlespacing*{\subsubsection}{0em}{1em}{-\parskip}%spacing text after sections from titlesec package

%when you want sectin/sub/subsub to be black, delete \color{ctublue}
\titleformat{\section}{\color{ctublue}\fontspec{Times New Roman}\fontsize{15}{15}\bfseries}{\thesection}{2.1em}{}%defining title sizes by word template
\titleformat{\subsection}{\color{ctublue}\fontspec{Times New Roman}\fontsize{14}{14}\bfseries}{\thesubsection}{1.53em}{}%defining title sizes by word template
\titleformat{\subsubsection}{\color{ctublue}\fontspec{Times New Roman}\fontsize{13}{13}\bfseries}{\thesubsubsection}{1em}{}%defining title sizes by word template


\usepackage{ctable}%imports xtable with booktabs
\usepackage{multicol}

\usepackage{listings}%for code environments - \begin{lstlisting}


\definecolor{codegreen}{rgb}{0,0.6,0}
\definecolor{codegray}{rgb}{0.5,0.5,0.5}
\definecolor{codepurple}{rgb}{0.58,0,0.82}
\definecolor{backcolour}{rgb}{0.95,0.95,0.92}


% solving problems with ) literal to be coded in lstlisting as it should be
\makeatletter
\patchcmd{\lsthk@SelectCharTable}{)}{`}{}{} 
\makeatother 

\lstdefinestyle{zakopal}{
    backgroundcolor=\color{codeblue},   
    commentstyle=\color{codegray},
    keywordstyle=\color{ctured},
    numberstyle=\tiny\color{codegray},
    stringstyle=\color{ctuorange},
    basicstyle=\ttfamily\small,
    breakatwhitespace=false,         
    breaklines=true,                 
    captionpos=b,                    
    keepspaces=true,                 
    numbers=left,                    
    numbersep=5pt,                  
    showspaces=false,                
    showstringspaces=false,
    showtabs=false,                  
    tabsize=2
}
\lstset{style=zakopal}
\renewcommand{\lstlistingname}{Kód}% renaming Listing -> Kód 
\renewcommand{\lstlistlistingname}{Seznam kódů}% renaming List of Listings -> Seznam kódů

\lstdefinelanguage{SCL}
{morekeywords={FUNCTION_BLOCK,BEGIN,NOT,END_FUNCTION_BLOCK,FUNCTION,VOID,VAR_INPUT,END_VAR,VAR_IN_OUT,IF,
THEN,END_IF,END_FUNCTION,BOOL,FALSE,TRUE},
sensitive=false,
morecomment=[l]{//},
morestring=[b]",
literate={;}{{\textcolor{ctuorange}{;}}}{1}
{:}{{\textcolor{ctuorange}{:}}}{1}
{)}{{\textcolor{ctuorange}{)}}}{1}
{(}{{\textcolor{ctuorange}{(}}}{1}
{=}{{\textcolor{ctuorange}{=}}}{1}
{,}{{\textcolor{ctuorange}{,}}}{1},} %basic SCL language for siemens defined%

\lstdefinelanguage{xdc}
{morekeywords={set_property, current_design, get_ports},
sensitive=false,
morecomment=[l]{\#}} %basic xdc file in Vivado syntax highlighting%

\lstdefinelanguage{xsct}
{morekeywords={xsct, hsi, open_hw_design, -createdts, -hw, -zocl, -platform-name, -overlay, -compile, -out, exit, -git-branch},
alsoletter={-},
sensitive=false,
morecomment=[l]{\#}} %xsct (Xilinx Software Command-Line Tools)%


\lstdefinelanguage{Text}
{morekeywords={},
alsoletter={-},
sensitive=false,
morecomment=[l]{//},
morecomment=[l]{\#}} %basic text%

\lstdefinelanguage{devicetree}
{morekeywords={chosen, bootargs, stdout-path, compatible, status},
alsoletter={-},
stringstyle=\color{ctuorange},
moredelim=[s][\color{ctuorange}]{"}{"},
sensitive=false,
morecomment=[l]{\#},
literate={\{}{{\textcolor{ctured}{\{}}}{1}
{\}}{{\textcolor{ctured}{\}}}}{1}
} %devicetree%

\lstdefinelanguage{json}
{morekeywords={},
upquote=true,
morestring=[b]",
stringstyle=\color{ctuorange},
moredelim=[s][\color{ctuorange}]{"}{"},
sensitive=false,
morecomment=[l]{\#},
literate=
     *{0}{{{\color{ctured}0}}}{1}
      {1}{{{\color{ctured}1}}}{1}
      {2}{{{\color{ctured}2}}}{1}
      {3}{{{\color{ctured}3}}}{1}
      {4}{{{\color{ctured}4}}}{1}
      {5}{{{\color{ctured}5}}}{1}
      {6}{{{\color{ctured}6}}}{1}
      {7}{{{\color{ctured}7}}}{1}
      {8}{{{\color{ctured}8}}}{1}
      {9}{{{\color{ctured}9}}}{1}
      {\{}{{{\color{ctured}{\{}}}}{1}
      {\}}{{{\color{ctured}{\}}}}}{1}
      {[}{{{\color{ctured}{[}}}}{1}
      {]}{{{\color{ctured}{]}}}}{1},
} %json%


%%change in previous commands 2.1 em , 1.53em and 1em to 1em to be easy indented not the same
\begin{document}
\fontspec{Times New Roman}

\counterwithin{figure}{section}%changing counter of figure, at each section the numbering resets
\counterwithin{table}{section}%changing counter of table, at each section the numbering resets
\counterwithin{equation}{section}%changing counter of equation, at each section the numbering resets

\counterwithin{lstlisting}{section}%counter of lstlist - codes, reseting at each section%


\renewcommand{\thefigure}{\thesection~-~\arabic{figure}}%defining style of countering
\renewcommand{\thetable}{\thesection~-~\arabic{table}}
\renewcommand{\theequation}{\thesection~-~\arabic{equation}}
\renewcommand{\thelstlisting}{\thesection~-~\arabic{lstlisting}}%delfining style for lstlisting codes, needs to be after begin document as previous renewcommand%

\renewcommand*{\cftsecdotsep}{1}  % use dots in the section entries and their step
\renewcommand*{\cftsubsecdotsep}{1}
\renewcommand*{\cftsubsubsecdotsep}{1}
\renewcommand*{\cftsecnumwidth}{4em} % increase space for Roman numerals
\renewcommand*{\cftsubsecnumwidth}{4em} %numbering width
\renewcommand*{\cftsubsubsecnumwidth}{4em} %numbering width
\renewcommand*{\cftsubsubsecindent}{0em}%no indent for subsubsection
\renewcommand*{\cftsubsecindent}{0em}%no indent for subsection
\renewcommand*{\cftsecindent}{0em}%no indent for subsection



\renewcommand*{\cftfigdotsep}{1}  % use dots in the figure entries and their step
\renewcommand*{\cftfignumwidth}{4em}
\renewcommand*{\cftfigindent}{0em}

\renewcommand*{\cfttabdotsep}{1}  % use dots in the figure entries and their step
\renewcommand*{\cfttabnumwidth}{4em}
\renewcommand*{\cfttabindent}{0em}

\renewcommand{\cftsecfont}{\fontspec{Times New Roman}\large \bfseries}
\renewcommand{\cftsubsecfont}{\fontspec{Times New Roman}}
\renewcommand{\cftsubsubsecfont}{\fontspec{Times New Roman}}

\renewcommand{\cftfigfont}{\fontspec{Times New Roman}}
\renewcommand{\cfttabfont}{\fontspec{Times New Roman}}

\renewcommand*\contentsname{\textcolor{ctublue}{\MakeTextUppercase{\fontspec{Times New Roman}Table of Contents}}}
\renewcommand{\listtablename}{{\fontspec{Times New Roman}\textcolor{ctublue}{\MakeTextUppercase{{List of tables}}}}}
\renewcommand{\listfigurename}{{\fontspec{Times New Roman}\textcolor{ctublue}{\MakeTextUppercase{{List of figures}}}}}




%\renewcommand{\thefigure}{\arabic{section}.\arabic{figure}}%changing figure name to be section.subsection. but do no reset
\setcounter{figure}{0}

\begin{titlepage}
	\begin{center}

\begin{figure}[H]
	\begin{center}
		\includegraphics[scale=1]{src/misc/symbol_cvut_konturova_verze.pdf}
	\end{center}
\end{figure}
	{\Large{\textbf{CZECH TECHNICAL UNIVERSITY IN PRAGUE}}}\\
	{\textbf{Faculty of Electrical Engineering}}\\
	{\textbf{Department of Electric Drives and Traction}}
	
	\vspace{3cm}
	
	
	{\Large\textbf{Name of the report}}
	
	\vspace{1cm}
	
	%{\Large\textbf{Possibilities of Using SoC Platform Processors for Controlling Electric Drives}}
	
	%\vspace{2cm}
	
	Technical report\\
	
	\end{center}
	
	\vspace{3cm}
	
	%\noindent Studijní program: Elektrotechnika, Energetika a Management\\
	%\noindent Studijní obor: Elektrické pohony
	
	\vspace{0.5cm}
	%\noindent Vedoucí práce: doc. Ing. Jan Bauer, Ph.D.
	
	\vfill
	
\begin{center}

	\large{\textbf{Petr Zakopal}}\\
	\large{\textbf{Prague 2023}}
	\end{center}
\end{titlepage}


\newpage
%\pagenumbering{arabic} to arabic page numbering
\pagenumbering{gobble} %disabling page numbering

\newpage


%%ZADÁNÍ PRÁCE
%verze pro TISK - jen s NEW PAGE


%ONLINE VERZE - se zadáním BEZ PODPISŮ
% online verze - odkomentovat následující dva řádky
%\null\newpage
%\includepdf[]{src/docs/zadani_bez_podpisu.pdf}

%\newpage
%\cleardoublepage
\null\newpage

\pagenumbering{Roman}
\setcounter{page}{2}%%3 NUTNO řešit dle zadání etc.

% \noindent \textcolor{ctublue}{{\Large{\textbf{\MakeTextUppercase{Prohlášení}}}}}\\
% 			Prohlašuji, že jsem předloženou práci vypracoval samostatně a že jsem uvedl veškeré použité informační zdroje v~souladu s~Metodickým pokynem o~dodržování etických principů při přípravě vysokoškolských závěrečných prací.\\
% 		\vspace{1.5cm}
		
	

% 	\noindent	V~Praze dne \rule{3.5cm}{0.4pt} \hspace{6.6cm}  \rule{4cm}{0.4pt}
	
% 	\hspace{12.65cm}Petr Zakopal


% 		\vspace{14cm}
		
% 	\noindent	\textcolor{ctublue}{{\Large{\textbf{\MakeTextUppercase{Poděkování}}}}}\\
% 	Tímto bych rád poděkoval vedoucímu této práce doc. Ing. Janu Bauerovi, Ph.D. za skvělé vedení práce a cenné rady při jejím vytváření. Dále bych rád poděkoval všem, kteří mě v~mém dosavadním studiu podporovali.
		


%%ABSTRAKT%%

% \newpage
% %\addcontentsline{toc}{section}{3\quad Abstrakt a klíčová slova}%Added citations to TOC%
% %\begin{comment}
% \begin{minipage}[t]{7.37cm}
% 		%\raggedright
% 	\textcolor{ctublue}{\Large{\textbf{\MakeTextUppercase{Abstrakt}}}}\\
	
% \end{minipage}%
% \hfill% --- important, otherwise it wont be so nice
% \begin{minipage}[t]{7.37cm}
% 		\textcolor{ctublue}{\Large{\textbf{\MakeTextUppercase{Abstract}}}}\\
		
% \end{minipage}
% %\end{comment}
% 	%\textcolor{ctublue}{\Large{\textbf{\MakeTextUppercase{Abstrakt}}}}\\

% 	%\textcolor{ctublue}{\Large{\textbf{\MakeTextUppercase{Abstract}}}}\\

% \newpage
\tableofcontents
\newpage%
\flushbottom %vyčištění stránky
\newpage
\vspace{0pt}
\listoffigures %seznam obrázků
\flushbottom %vyčištění stránky
\newpage
\listoftables
\flushbottom
\newpage


\pagenumbering{arabic} %to arabic page numbering - enabling page numbering after gobble which disabled page numbering
\pagenumbering{gobble}
\null\newpage
% \null\newpage %PŘI VERZI ONLINE
\setcounter{page}{1}
\pagenumbering{arabic}
\fontspec{Times New Roman}

\section{Introduction}
\gls{abbreviation:soc} \gls{symbol:Pn}.

\flushbottom %vyčištění stránky
\newpage
%konec úvodu


		
%závěr
\newpage
\addcontentsline{toc}{section}{\numberline{}Conclusion} 
\section*{Conclusion}


\flushbottom %vyčištění stránky

%konec závěru

\newpage
\setmonofont{Times New Roman}
\printbibliography[title={{References}}]	
\nocite{*}
\setmonofont{Courier}
\addcontentsline{toc}{section}{\numberline{}References} %Added citations to TOC%
	\appendix
	\titleformat{\section}{\color{ctublue}\fontspec{Times New Roman}\fontsize{15}{15}\bfseries}{Appendix \thesection:}{2.1em}{}
	\begin{appendices}
	\section{List of symbols and abbreviations}

		\printglossary[type=abbreviationslist, style = myStyleAbbreviations]

		\fbar
		\newpage
		\printglossary[type=symbolslist, style =  myStyleSymbols]

	\end{appendices}
\end{document}
